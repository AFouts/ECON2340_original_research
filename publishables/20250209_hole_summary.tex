\documentclass{article}
\usepackage[left=1.5cm, right=1.5cm, top=1.5cm, bottom=2cm]{geometry}
\usepackage[utf8]{inputenc}
\usepackage{amsmath,amsthm,stmaryrd,amssymb,bbm,amsfonts,amstext,graphicx,multicol,array}
\usepackage{subfigure}
\usepackage{xcolor}
\usepackage{enumitem}
\usepackage{indentfirst}
\usepackage{caption}
\usepackage{pdfpages}
\usepackage[numbers]{natbib} % Use natbib with numerical citations
\usepackage{hyperref}
\usepackage{float}
\usepackage{booktabs}
\usepackage{graphics, graphicx}
\usepackage{booktabs}
\usepackage{adjustbox}

\hypersetup{
    colorlinks=true,
    linkcolor=blue,
    urlcolor =blue,
    citecolor = blue
}


\setlength{\parindent}{2em}
\setlength{\parskip}{1em}
\renewcommand{\baselinestretch}{1.5}

\title{ECON2340 Spring 2025\\Original Research\\HOLE Chapter Summary\\Chapter 6 Job search, unemployment insurance, and active labor market policies\\Thomas Le Barbanchon, Johannes Schmieder, and Andrea Weber}
\author{Adrien Foutelet}
\date{February 2025}

\begin{document}

\maketitle

Modern unemployment insurance (UI) systems typically take one of three forms: tax-financed national initiatives, contribution-based social insurance, or a combination of both. These systems are often supplemented by severance pay (SP) schemes. Governments may contribute to UI through passive spending and support active labor market policies (ALMP) through active spending.

\section{Micro-foundations of job search among the unemployed}

A microeconomic theory of job search includes profit-maximizing firms (labor demand), utility-maximizing individuals (labor supply), and labor market frictions to account for involuntary unemployment, all considered within a partial equilibrium framework. Frictions include imperfect knowledge of available jobs, reservation wages, and job search effort. The basic predictions are that, within homogeneous worker types, increasing the generosity of UI benefits decreases the job-finding rate and weakly increases reservation wages and reemployment wages, conditional on unemployment duration. With multiple worker types, UI generosity increases unemployment duration, but due to dynamic selection, reemployment wages do not necessarily increase for all unemployment durations because of the changing composition of the unemployed (e.g., skill depreciation).

The main stylized empirical findings that hold across time and space on this matter are primarily derived from regression discontinuity designs (RDD). They are:

\begin{itemize}
    \item An increase in UI generosity leads to longer unemployment durations.
    \item An increase in potential benefit duration (PBD) lowers the periodic job-finding hazard.
    \item UI exhaustion leads to a spike in the job-finding hazard.
    \item Longer unemployment duration is associated with lower reemployment wages.
    \item Changes in PBD have a moderate and ambiguous effect on reemployment wages.
\end{itemize}

Extending the multiple-type framework of job search models allows for a better fit with empirical patterns observed under UI policies. However, multiple functional forms, the nature of heterogeneity, and different microfoundations (i.e., sources of job-finding hazards) can all explain the same empirical patterns, leading to underidentification. Distinguishing between these models is challenging and often requires complementary data sources. Typically, administrative data alone is insufficient for this purpose, necessitating the use of high-frequency panel survey data, online job platform data, consumption data, etc.

Survey data on search effort include the American Time Use Survey (Krueger and Mueller, 2010, 2012). However, these datasets are limited because they are relatively small when conditioning on unemployment and are cross-sectional. Panel data tracking individuals throughout their unemployment spell have also been collected and studied. Krueger and Mueller (2011) conducted an online study with 63,000 individuals over two years. DellaVigna et al. (2022) collected data via text messages over 18 weeks per cohort. These studies confirmed the stylized facts, providing more precise insights into job search behavior, particularly the decline in effort over time, the spike in search intensity at UI exhaustion, and the deterioration of job seekers' well-being. Data from search platforms also support these findings. Notably, Marinescu (2017) used data from careerbuilder.com and found that PBD extensions do not affect labor market tightness. The decline in search effort over time remains a debated finding, as it may be counterbalanced by unemployed individuals switching to different search platforms. Additionally, UI claim audit data from the U.S. (Massenkoff, 2023) suggests no impact of UI on search effort.

Regarding reservation wages, which are difficult to measure, the literature often relies on target wages inferred from job applications. Survey data from Krueger and Mueller (2016) indicate that reported reservation wages are, on average, at least as high as pre-unemployment wages, exhibit high variance, and are inelastic to benefit levels and unemployment duration, except for older and higher-earning individuals. Le Barbanchon et al. (2019) used data from the French UI system and found no effect of PBD on reservation wages. Other studies suggest that target wages decrease slightly over time.

In terms of consumption, Gruber (1997) found a significant drop in food consumption at the onset of unemployment using the Panel Study of Income Dynamics, with UI benefits mitigating this decline. Ganong and Noel (2019) used high-frequency JP Morgan Chase data to document a 6\% drop in consumption at the start of unemployment, followed by a 1\% decline per month, leading to a total decrease of 25\% by UI exhaustion.

The job search model itself has undergone significant refinements in recent years. It is now well-established that search intensity responds strongly to UI generosity, whereas job selectivity does not. Moderate evidence suggests that reemployment wages depend on unemployment duration, influenced by skill depreciation and declining reservation and target wages. Search effort plays a more significant role in shaping job search outcomes than reservation wages. Strong evidence indicates that job-finding rates depend on unemployment duration, primarily due to dynamic selection rather than search effort or reservation wages. Furthermore, present bias is evident, leading to insufficient job search effort (averaging 60–90 minutes per day). 

The role of reference dependence (i.e., comparing job offers to past wages) remains ambiguous. It is unclear whether job seekers overestimate their job-finding probability, underestimate the returns to search, or have a locus of control that affects their job search effort and outcomes. Future research should further explore the impact of mental health during unemployment (Ahammer and Pachham, 2023; Koszegi et al., 2022).

\section{Design of UI Policy}

The standard framework for optimal UI was introduced by Baily (1978) and extended by Chetty (2006, 2008). The Baily-Chetty formula provides a direct mapping between theoretical welfare effects and their empirical counterparts, justifying the computation of the elasticity of unemployment duration with respect to benefit generosity and the marginal utility change from employment to unemployment. A dynamic version of the model allows for the consideration of PBD (Schmieder and van Wachter, 2016). In this version, a higher share of unemployed workers exhausting their benefits leads to lower tax revenue, increasing the tax burden on employed individuals. As a result, the sufficient statistics for optimal UI become the behavioral costs (elasticity of expected duration of covered unemployment with respect to benefit level and the elasticity of non-employment duration) and the consumption-smoothing benefits (the social value of UI changes).

The main identification challenge in estimating causal elasticities is that unemployed workers may self-select into different UI categories based on unobservables (e.g., more experienced workers may qualify for longer PBD). When PBD is a deterministic function of past work experience with discontinuous jumps, RDD provides an appropriate identification strategy (Card et al., 2007). Regression kink designs (RKD) are also widely used (Card et al., 2015a, 2015b; Landais, 2015), as caps on benefit levels generate kinks in the relationship between previous wages and benefits. Difference-in-differences (DiD) approaches leverage exogenous policy reforms affecting only a subset of workers, providing a natural control group. However, UI reforms are rarely exogenous, weakening most DiD estimates. In the U.S., UI is countercyclical by design, as Extended Benefits and Emergency Unemployment Compensation programs activate beyond certain unemployment thresholds. To account for this, trigger designs control for a flexible parametric function of unemployment when regressing unemployment duration on PBD, with causal identification obtained from the discontinuous jump in PBD at policy activation thresholds (Rothstein, 2011). Across the literature, the average elasticity of unemployment duration with respect to PBD is estimated at 0.49, while the elasticity with respect to the replacement rate is 0.40. Using a UI tax rate of 3\%, the median behavioral cost of transferring an additional dollar of UI benefits is estimated at USD 0.35.

The social value of more generous UI can be quantified using a consumption-based approach (Gruber, 1997), which suggests that the marginal value of a one-dollar transfer in unemployment benefits is USD 0.13. Chetty (2008) develops an alternative approach based on the liquidity-to-moral hazard ratio, leveraging policy variation (e.g., changes in severance payments) to infer the social value of UI from behavioral responses such as savings. He decomposes the search response into substitution and income effects, with the pure moral hazard cost estimated as the difference between the total search effect and the income effect. A third approach, the marginal-propensity-to-consume (MPC) method, combines the two (Landais and Spinnewijn, 2021). A less explored but theoretically straightforward method is revealed preferences, which infers the social value of UI from the price individuals are willing to pay for extra coverage. Overall, estimates of the social value of UI vary widely across identification strategies, with recent estimates—robust to assumptions about risk aversion—yielding high values.

Returning to the Baily-Chetty formula, by plugging in the three key estimates, one can determine whether UI is at its optimal level based on whether the marginal welfare effect of transferring one dollar to an unemployed worker is positive or negative. However, this framework does not guide policymakers on whether UI should be prioritized over other policy instruments. To address this, the Marginal Value of Public Funds (MVPF) framework (Hendren and Sprung-Keyser, 2020) is used, which does not require the government to balance the budget constraint. Based on calibrated estimates from the literature, European MVPFs for increasing UI benefits range between 0.24 and 0.99, while U.S. MVPFs range between 0.51 and 1.18. In Europe, the existing policy mix between UI benefits and PBD appears optimal, whereas in the U.S., welfare could be improved by increasing PBD spending while reducing benefit levels. However, further investigation is needed, as the number of available sufficient statistics estimates remains limited. Formally, because UI exhaustees benefiting from PBD extensions tend to have lower potential wages, redistribution-oriented policy mixes would favor extending PBD.

Recent studies, such as Schmieder et al. (2016), suggest that more generous UI policies reduce wages, imposing second-order behavioral costs through higher taxes. The impact of UI eligibility on job separation remains ambiguous, with studies finding either no effect or positive effects (e.g., Van Doornik et al., 2023). The effect appears to be more significant for senior workers (e.g., Jäger et al., 2023). Lusher et al. (2022) found that an 18-week PBD extension during the Great Recession in the U.S. led to a 2\% decrease in cashier scanning speed, suggesting modest effects on work effort.

In practice, UI benefit designs are time-varying. Shavell and Weiss (1979) provide theoretical reasons for making benefits decline over time, although an initially increasing trajectory might be optimal when workers have some initial wealth at job loss. Empirical studies (e.g., Kolsrud et al., 2018) highlight the importance of introducing an endogenous wedge between consumption and benefits.

Whether UI benefits and PBD should be procyclical (Kroft and Notowidigdo, 2016) or countercyclical (Schmieder et al., 2012) remains an open question. Microeconomic evidence suggests that the behavioral costs of UI are lower during recessions.

The interaction between UI and other programs, such as disability insurance (DI), remains unclear, and further research is needed on the positive externalities of UI on crime and health.

On the macroeconomic side, extending the Baily-Chetty framework to a general equilibrium (GE) setting shows that UI policies may influence job-finding rates among uncovered job seekers and affect job creation. To account for these effects, models incorporate labor market tightness into individual job-finding rates and evaluate whether UI programs improve social welfare by pushing tightness toward its efficient level (Landais et al., 2018). To quantify macro effects, research designs have been developed to mimic market-level randomization (e.g., among job seekers, vacancies, and wages). Recent empirical evidence (e.g., Boone et al., 2012) suggests that the macro elasticity of unemployment with respect to UI—the total response of unemployment to UI changes, accounting for general equilibrium effects—is not larger than its micro counterpart (the change in unemployment due to the search effort response, holding tightness constant), and may even be smaller. This suggests that macroeconomic spillovers from UI policies are limited.

\section{Active Labor Market Policies}

ALMPs aim to reduce moral hazard from benefit receipt by imposing search requirements and monitoring job search efforts. They also seek to improve the efficiency of individual job search and accelerate unemployed workers' return to employment by providing counseling. Finally, they aim to reduce skill mismatches in the labor market by offering training programs to low-wage and unemployed workers, enabling them to access better-paid jobs and improve their labor market outcomes. Card et al. (2018) consider five types: job search assistance (JSA), training programs, employment subsidies in the private sector, public sector employment programs, and multi-component programs combining multiple interventions. Their meta-analysis shows that short-run program effects tend to improve in the medium and long run. The time profile of program effects varies by type: job search assistance programs exhibit stable effects over different horizons, whereas programs with a human capital component tend to show improving effects over time. There is heterogeneity in program effects across participant groups, highlighting the potential benefits of matching specific job seekers to tailored interventions. Program effectiveness also depends on cyclical conditions, with larger impacts observed during periods of low GDP growth or high unemployment.

A number of recent studies provide insights into the role of caseworkers in ALMP programs by exploiting data on job-seeker and caseworker matches and using RCTs with controlled assignments of job seekers to caseworkers. Michaelides and Mueser (2020) evaluate four programs implemented in three U.S. states through an experimental design. Potential participants were randomly assigned to receive letters inviting them to meetings with a caseworker, where their eligibility status was assessed. Non-eligible benefit recipients were disqualified, while eligible individuals received counseling services, job search information, and referrals to training programs. The programs reduced unemployment benefit durations either because individuals did not attend the meeting or because they were disqualified after assessment. Programs incorporating counseling components increased employment and earnings in the first year after program assignment. Manoli et al. (2018) show that these effects persisted for up to eight years. Cederlöf et al. (2021) estimate the value added by caseworkers in Sweden with respect to job finding and job quality, showing that caseworker experience and job market expertise substantially influence search outcomes, long-run earnings, and employment stability.

Recent evaluations of ALMPs have focused on specific disadvantaged groups. Bobonis et al. (2022) assess the long-term effects of the Self-Sufficiency Project Regular (offering earnings supplements) and Plus (offering earnings supplements and intensive employment support services) in Canada. The Plus program led to sustained earnings gains, whereas gains from the Regular program faded quickly after the supplements expired. Participants in the Plus program experienced increased full-time employment and reduced welfare dependency. The support services facilitated career advancement into better-paid and more stable jobs. Survey evidence also suggests improvements in non-cognitive skills and measures of grit. Other commonly targeted groups include immigrants, refugees, and youth.

ALMP design increasingly incorporates demand-side considerations. Identifying occupations in high demand and appropriate wage levels typically involves labor market surveys. ALMPs are also being adapted to developing countries, where labor market frictions on both the employer and worker sides hinder efficient job matching. Alfonsi et al. (2020) analyze general equilibrium effects and conclude that vocational training for young workers is more effective than firm-based training incentives. A key determinant of labor market success for vocational training participants is a recognized skill certificate, which allows workers to credibly signal their skills and facilitates career mobility.

Spillover effects from treatment to control groups can pose challenges to identifying causal effects. Crépon et al. (2013) evaluate a job search assistance program for young unemployed university graduates in France, implemented across multiple local labor markets, using a double-randomization strategy. In the first step, treatment intensities—determining the share of job seekers receiving JSA—were randomly assigned across labor markets. In the second step, eligible job seekers were randomly assigned to the JSA program within each market according to its assigned treatment intensity. This design allows for within-region comparisons of treated and control individuals and cross-region comparisons of different treatment intensities. If individuals in control groups within high-intensity regions systematically experience worse outcomes than those in untreated regions, this indicates the presence of spillover effects. The study finds evidence of substantial displacement effects, where more jobs were lost than gained, leading to an overall net negative employment impact. Other notable studies on spillovers include Cheung et al. (2023), Altmann et al. (2022), and Belot et al. (2019).

Welfare analyses of ALMPs remain relatively recent and heterogeneous in terms of the outcomes considered. The use of the Marginal Value of Public Funds (MVPF) framework (Hendren and Sprung-Keyser, 2020) is not yet the standard. In addition to earnings, studies increasingly consider broader outcomes such as cognitive and non-cognitive skills, health, mortality, well-being, life and job satisfaction, social well-being, job search strategies, and search effort. The mechanisms through which programs operate appear to include targeting well-selected groups, issuing certificates, appropriate program timing, and the acquisition of non-cognitive skills. A deeper understanding of these mechanisms requires further development of structural models, which could facilitate counterfactual policy simulations and allow for a rigorous evaluation of ALMPs’ general equilibrium effects.


\end{document}
